% Chapter 2
\chapter[XÂY DỰNG HỆ THỐNG GỢI Ý PHIM BẰNG THUẬT TOÁN ITEM-BASED COLLABORATIVE FILTERING]
 {\LARGE XÂY DỰNG HỆ THỐNG GỢI Ý PHIM BẰNG THUẬT TOÁN ITEM-BASED COLLABORATIVE FILTERING}

% Section 1 Chapter 2
\section{Giới thiệu thuật toán}
Thuật toán Item - based Collaborative Filtering (CF) là một phương pháp
phổ biến trong hệ thống gợi ý, được sử dụng để đưa ra các đề xuất cá nhân
hóa cho người dùng.
\vspace{0.5cm}
\newline
Thuật toán này phân tích ma trận người dùng - sản phẩm
để xác định mối quan hệ giữa các sản phẩm khác nhau và sau đó sử dụng các
mối quan hệ này để tính toán gián tiếp các đề xuất cho người dùng.
\vspace{0.5cm}
\newline
Phương pháp này giúp cải thiện hiệu quả và chất lượng của hệ thống
gợi ý so với các thuật toán CF dựa trên người dùng.

% Section 2 Chapter 2
\section{Triển khai thuật toán}
Triển khai thuật toán Item - based Collaborative Filtering
bao gồm các bước chính sau đây:

\subsection*{Bước 1: Xây dựng ma trận đồng xuất hiện}
Tạo ra ma trận đồng xuất hiện C kích thước n x n, trong đó
C[i][j] là số lượng người dùng đã đánh giá cả sản phẩm i và j.

\subsection*{Bước 2: Chuẩn hóa ma trận}
Tạo ra ma trận chuẩn hóa S kích thước n x n, trong đó: \\
$$S[i][j] = \frac{C[i][j]}{\sum_{m \in M}C[m][j]}$$ \\
Trong đó M là tập hợp tất cả các sản phẩm.

\subsection*{Bước 3: Xây dựng ma trận đánh giá}
Tạo ra ma trận đánh giá R kích thước n x m, trong đó: \\
$$R[i][j] = \sum_{m \in M}S[i][m] \times U[m][j]$$ \\
Trong đó M là tập hợp tất cả các sản phẩm và thay $U[i][j]=?$ thành $U[i][j]=0$

\subsection*{Bước 4: Triển khai với Hadoop MapReduce}
Để triển khai thuật toán này trên một tập dữ liệu lớn, chúng ta sử dụng
Hadoop MapReduce để phân chia và xử lý dữ liệu song song. Quá trình triển khai
bao gồm các bước sau:
\pagebreak
\begin{itemize}
    \item \textbf{Mapper}: Đọc vào các cặp key/value biểu diễn dữ liệu
          người dùng - sản phẩm, tính toán các cặp sản phẩm tương tự và
          lưu trữ các kết quả trung gian.
    \item \textbf{Reducer}: Tập hợp các kết quả từ Mapper, tính toán
          độ tương đồng giữa các sản phẩm và tổng hợp các dự đoán cuối cùng.
\end{itemize}

% Section 3 Chapter 2
\raggedright
\section{Ví dụ minh họa thuật toán}
Để minh họa cho quá trình hoạt động của thuật toán CF dựa trên Item,
chúng ta xem xét một ví dụ đơn giản:\\
\vspace{0.5cm}
\textbf{Dữ liệu đầu vào}: Ma trận người dùng -
sản phẩm với các đánh giá từ 1 - 5. (? là sản phẩm mà người dùng đó chưa đánh giá)\\
\renewcommand{\arraystretch}{2}
{\centering
    \begin{tabular}{ |c|c|c|c|c| }
        \hline
                     & Sản phẩm 1 & Sản phẩm 2 & Sản phẩm 3 & Sản phẩm 4 \\
        \hline
        Người dùng A & 5          & 3          & ?          & 4          \\
        \hline
        Người dùng B & 3          & ?          & 2          & 3          \\
        \hline
        Người dùng C & 4          & ?          & 4          & ?          \\
        \hline
        Người dùng D & ?          & 3          & ?          & 5          \\
        \hline
        Người dùng E & ?          & ?          & ?          & ?          \\
        \hline
    \end{tabular}
    \par}

\subsection*{Bước 1: Xây dựng ma trận đồng xuất hiện}
Dựa trên dữ liệu đầu vào, chúng ta xây dựng ma trận đồng xuất hiện C như sau:
(Ví dụ ta thấy rằng có 2 người cùng đánh giá sản phẩm 1 và 3 là Người dùng B và C
nên $C[1][3] = C[3][1] = 2$)\\
\vspace{0.2cm}
{\centering
    \begin{tabular}{ |c|c|c|c|c| }
        \hline
                   & Sản phẩm 1 & Sản phẩm 2 & Sản phẩm 3 & Sản phẩm 4 \\
        \hline
        Sản phẩm 1 & 3          & 1          & 2          & 2          \\
        \hline
        Sản phẩm 2 & 1          & 2          & 0          & 2          \\
        \hline
        Sản phẩm 3 & 2          & 0          & 2          & 1          \\
        \hline
        Sản phẩm 4 & 2          & 2          & 1          & 3          \\
        \hline
    \end{tabular}
    \par}

\subsection*{Bước 2: Chuẩn hóa ma trận}
Dựa trên ma trận C, chúng ta chuẩn hóa thành ma trận S như sau:
(Tổng cột sản phẩm 1 là 8 nên $S[1][1] = \frac{C[1][1]}{8} = \frac{3}{8} = 0.375$)\\
\vspace{0.2cm}
{\centering
    \begin{tabular}{ |c|c|c|c|c| }
        \hline
                   & Sản phẩm 1 & Sản phẩm 2 & Sản phẩm 3 & Sản phẩm 4 \\
        \hline
        Sản phẩm 1 & 0.375      & 0.2        & 0.4        & 0.25       \\
        \hline
        Sản phẩm 2 & 0.125      & 0.4        & 0          & 0.25       \\
        \hline
        Sản phẩm 3 & 0.25       & 0          & 0.4        & 0.125      \\
        \hline
        Sản phẩm 4 & 0.25       & 0.4        & 0.2        & 0.375      \\
        \hline
    \end{tabular}
    \par}

\pagebreak
\subsection*{Bước 3: Xây dựng ma trận đánh giá}
Dựa trên ma trận S và ma trận đầu vào U, ta tính được ma trận R như sau:
(Ví dụ ta tính đánh giá của người dùng A cho sản phẩm 3 là
$R[1][3] = 5 \times 0.4 + 3 \times 0 + 0 \times 0.4 + 4 \times 0.2 = 2.8$)\\
\vspace{0.5cm}
{\centering
\begin{tabular}{ |c|c|c|c|c| }
    \hline
                 & Sản phẩm 1 & Sản phẩm 2 & Sản phẩm 3 & Sản phẩm 4 \\
    \hline
    Người dùng A & 3.25       & 3.8        & 2.8        & 3.5        \\
    \hline
    Người dùng B & 2.375      & 1.8        & 2.6        & 2.125      \\
    \hline
    Người dùng C & 2.5        & 0.8        & 3.2        & 1.5        \\
    \hline
    Người dùng D & 1.625      & 3.2        & 1.0        & 2.625      \\
    \hline
    Người dùng E & ?          & ?          & ?          & ?          \\
    \hline
\end{tabular}
\par}
\vspace{0.2cm}
Người dùng E chưa đánh giá sản phẩm nào nên ta không thể tính được đánh giá của người đó.

\subsection*{Kết luận}
Dựa trên tính toán ở trên, dự đoán đánh giá của người dùng A cho Sản phẩm 3 là khoảng 2.8.
Qua ví dụ minh họa này, chúng ta thấy rằng:
\begin{itemize}
    \item Thuật toán CF dựa trên Item giúp xác định mối quan hệ
          giữa các sản phẩm dựa trên đánh giá của người dùng.
    \item Việc tính toán độ tương đồng giữa các sản phẩm và sau đó
          sử dụng nó để tính toán dự đoán đánh giá giúp cải thiện chất lượng gợi ý.
    \item Thuật toán này có thể mở rộng để xử lý dữ liệu lớn bằng cách
          sử dụng Hadoop MapReduce, giúp phân chia và xử lý dữ liệu song song,
          tối ưu hóa hiệu suất và tốc độ xử lý.
\end{itemize}